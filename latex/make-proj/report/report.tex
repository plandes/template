\documentclass[12pt]{article}

%% Imports
%
\usepackage{${project}}
%\usepackage[usesidenotes]{${project}}
\usepackage{fullpage}
\usepackage{zenmath}
\usepackage{zenreport}


%% Configure
%
% document type (below title)
\renewcommand{\doctype}{Report}
% remove date
\date{}

% enable default hyperlink behavior
\zzhrefinit%
% disable default hyperlink behavior
%\zzhrefdisable\relax

% disable acronym links, hover over features, and italics
%\zzacrodisablelinks%

% configure the report document class
\zrreportinit%
%\zrreportinitacademic%
\zrrpage%
\zrrfontset%

% options: alphabetic, authoryear (default), numeric-comp, alphabetic-verb, alphabetic-comp
% https://www.overleaf.com/learn/latex/Biblatex_citation_styles
\zzaddbiblatexstyle[numeric]{${project}.bib}

% cite commands
% \zzaddcommands%


%% Doc
%
\begin{document}
\zrreptitle%
%\tableofcontents

\begin{abstract}
  Latent semantic analysis (LSA) is a technique in natural language processing,
  in particular distributional semantics, of analyzing relationships between a
  set of documents and the terms they contain by producing a set of concepts
  related to the documents and terms.
\end{abstract}


#[[\zrsec{Introduction}]]#
#[[]]#
#[[The \citeauthor{\deercite} method uses term document matrices~\cite{\deercite}.]]#
#[[An example implementation is provided with the \lsacode.  The \lsacode\ is freely available.]]#
#[[Also see \zzhref{https://en.wikipedia.org/wiki/Latent\_semantic\_analysis}.]]#
#[[]]#
#[[]]#
#[[]]#
#[[\zrsubsec{Complex Entity Matching Method}]]#
#[[]]#
#[[To better match entities in cases where the author uses several aliases one]]#
#[[must use the writing style for this resolution.  One way we classify]]#
#[[an author's writing style using \nlp\ methods are with LSA.]]#
#[[]]#
#[[This algorithm is outlined below:]]#
#[[]]#
#[[\begin{enumerate}]]#
#[[\item Create a term frequency matrix.  In this step we count the number of]]#
#[[  times each term appears in all documents using the rows as the terms and the]]#
#[[  columns as the document.  For us a document is a message board entry, tweet,]]#
#[[  a Facebook post, etc.  Call this matrix $X$.]]#
#[[]]#
#[[  See \zzfigref{lsa} for an example initial matrix.]]#
#[[]]#
#[[\zzfigure{1.7in}{lsa}{Initial Example Matrix.}]]#
#[[]]#
#[[\item Optionally weight each cell in $X$ with the TF/IDF frequency as an]]#
#[[  optimization step.]]#
#[[]]#
#[[\svd{4in}{3}{sorted matrices}]]#
#[[]]#
#[[\item Perform SVD.  This is a special kind of eigendecomposition on $X$ by]]#
#[[  separating such that:]]#
#[[\[X = U \, \Sigma V\zmtrans\]]]#
#[[  where $X$ is our term frequency matrix, $U$ and $V$ are the singular vectors,]]#
#[[  which are a set of orthogonal eigenvectors.]]#
#[[  $\Sigma$ is a diagonal matrix composed of the eigenvalues of $U$ and $V$.]]#
#[[  See \zzfigref{lsa2}.]]#
#[[]]#
#[[\svd{\textwidth}{2}{initial matrix}]]#
#[[]]#
#[[\item Sort vectors in $U$ and $V$ using the ordering of the descending {\it]]#
#[[    eiganvalues} from $\Sigma$.  See \zzfigref{lsa3}.]]#
#[[]]#
#[[\item First, $k$ is set as the number of clusters a priori and represents the]]#
#[[  number of authors we choose to cluster together.  Then create a new latent]]#
#[[  word embedding vector set with:]]#
#[[  \[\hat{X}_k = \sum_{i=1}^{k} U_{ik} \Sigma_{ii} V_{ki}\zmtrans\]]]#
#[[  using only the first $k$ rows/columns to {\it truncate} the original SVD]]#
#[[  matrix.]]#
#[[]]#
#[[\svd{\textwidth}{4}{truncated matrix}]]#
#[[]]#
#[[\item Create matrix based on the cosine similarity:]]#
#[[]]#
#[[\[ \mathbf{S} = [S_{i,j}] \]]]#
#[[where]]#
#[[\[S_{i,j} = \textrm{sim}(\hat{X}_{k,i}, \hat{X}_{k,j})\]]]#
#[[]]#
#[[\item Run clustering on $S$ (k-means, k-centering, etc) on $\mathbf{S}$.]]#
#[[\end{enumerate}]]#



%\clearpage
%\pagebreak
\medskip
\printbibliography%

\printacronyms[include-classes=abbrev, name=Abbreviations]%

\printacronyms[include-classes=def, name=Definitions]%

% this needs \makeindex in the preamble (${project}.sty) and tex-index in makefile
%\printindex%

\end{document}
